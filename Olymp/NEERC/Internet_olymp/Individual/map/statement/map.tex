\begin{problem}{Карта}{map.in}{map.out}{2 секунды}{256 мегабайт}

%Автор задачи: Колян Ведро
%Колян-ведро - морозь табло!!!
%Автор условия: Демид Кучеренко

\setlength{\epigraphwidth}{.40\textwidth}
\epigraph{
	Даже самый последний матрос знает, что мы едем искать сокровища. Не нравится мне всё это!
}
{Капитан Смоллетт}


В далеком 1744 году во время долгого плавания в руки капитана 
Александра Смоллетта попала древняя карта с указанием местонахождения 
сокровищ. Однако расшифровать ее содержание было не так уж и просто. 


Команда Александра Смоллетта догадалась, что сокровища находятся на $x$ 
шагов восточнее красного креста, однако определить значение числа $х$
она не смогла. По возвращению на материк Александр Смоллетт решил обратиться 
за помощью в расшифровке послания к знакомому мудрецу. Мудрец поведал,
что данное послание таит за собой некоторое число. Для вычисления этого 
числа необходимо было удалить все пробелы между словами, а потом посчитать 
количество способов вычеркнуть все буквы кроме трех так, чтобы полученное слово 
из трех букв одинаково читалось слева направо и справа налево.

Александр Смоллетт догадывался, что число, зашифрованное в послании, и есть 
число $x$. Однако, вычислить это число у него не получилось.

После смерти капитана карта была безнадежно утеряна до тех пор, пока не оказалась 
в ваших руках. Вы уже знаете все секреты, осталось только вычислить число $x$.
 
\InputFile
В единственной строке входного файла дано послание, написанное на карте. Длина 
послания не превышает $3 \cdot 10^5$. Гарантируется, что послание может содержать только строчные 
буквы английского алфавита и пробелы. Также гарантируется, что послание не пусто. Послание 
не может начинаться с пробела или заканчиваться им.

\OutputFile
Выведите одно число $x$~--- количество сособов вычеркнуть из послания все буквы кроме трех так, 
чтобы оставшееся слово одинаково читалось слева направо и справа налево.

\Examples
\begin{example}%
\exmp{
treasure
}{
8
}%
\exmp{
you will never find the treasure
}{
146
}%
\end{example}

\Note
Решения, работающие в случаях, в которых длина послания не превосходит $200$, будут оцениваться в $30$ баллов.

Решения, работающие в случаях, в которых длина послания не превосходит $5000$, будут оцениваться в $60$ баллов.

\end{problem}
