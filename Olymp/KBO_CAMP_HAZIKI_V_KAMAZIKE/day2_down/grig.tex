\begin{problem}{Кузнечик}{stdin}{stdout}{1 секунда}{64 мегабайта}

У одного из преподавателей параллели С в комнате живёт кузнечик, который очень любит прыгать по клетчатой одномерной доске. Длина доски~--- $N$ клеток. К его сожалению он умеет прыгать только на $1$, $2$, $\ldots$, $k$ клеток вперёд. 

Однажды преподавателям стало интересно, сколькими способами кузнечик можетдопрыгать из первой клетки до последней. Помогите им ответить на этот вопрос.


\InputFile
В первой и единственной строке входного файла записано два целых числа~--- $N$ и $k$ ($1 \le N \le 30, 1 \le k \le 10)$.


\OutputFile
Выведите одно число~--- количество способов, которыми кузнечик может допрыгать из первой клетки до последней.

\Examples

\begin{example}
\exmp{8 2
}{21}%
\end{example}

\end{problem}
