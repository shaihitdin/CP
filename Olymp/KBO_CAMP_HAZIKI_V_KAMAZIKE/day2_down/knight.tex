\begin{problem}{Черепахоконь}{stdin}{stdout}{1 секунда}{64 мегабайта}

Дана клетчатая доска размером $N \times M$ в каждой клетке которой записано натуральное число. В верхнем левом углу доски сидит черепашка. Черепашка умеет делать ход конём по направлению вниз и вправо. То есть либо перемещаться на одну клетку вправо и на две вниз, либо на одну клетку вниз и на две вправо. Помогите черепашке добраться в правый нижний угол доски, собрав максимальную сумму чисел. Считается, что черепашка собирает только те числа, на которых завершает ход, а не все, по которым проползает. 

\InputFile
В первой строке входного файла два целых числа $N$ и $M$ ($1 \le N, M \le 100$), задающие размеры доски. Далее следуют числа, записанные на доске~--- $N$ строк по $M$ положительных чисел, не превышающих $10\,000$, в каждой.



\OutputFile
Выведите одно число, равное искомой максимальной сумме, либо -1, если черепашка не может добраться до правого нижнего угла.

\Examples

\begin{example}
\exmp{2 3
3 2 7
1 9 5
}{8}%
\end{example}

\end{problem}
